\documentclass[../dissertation.tex]{subfiles}

\begin{document}

\begin{bibunit}

\chapter{Conclusion}

\section{Summary of Findings}

We found that there are several potential structural fingerprints of parasites.
The most significant property that differs between the community of free living
carnivores and the community of parasites is consumer-resource clustering.  A
parasite has on average twice as many links from its consumers to its
resources. Consumer clustering and trophic level exhibited only marginally
significant differences between the free-living carnivores and parasites. The
overall fingerprint is subtle but it does exist.  However, the other 9
properties were clearly not significant, suggesting that differences between
free livers and parasites are the exception, not the rule. The observed
differences between free livers and parasites were robust to decreases in
trophic resolution and so are unlikely to be a result of sampling biases.

We also found that parasites can be differentiated from carnivores to a high
degree of accuracy using classification trees with only 4 binary splits. The
ability to distinguish parasite from carnivore was also robust to decreases in
trophic resolution. We found that consumer clustering and resource-consumer
clustering were the most commonly used variables across all levels of
clustering. Trophic level was also a frequently used property in defining
binary splits. Parasites are distinguished most by the way in which
their neighbors interact with each other.

The inverse niche model produced ensembles of food webs that matched empirical
food web properties better than the niche model with random parasites. The use
of an inverted generality hierarchy combined with matching the number of links
between and within parasite and free living communities were both important to
achieving this improvement. The inverse niche model was also better at matching
properties at decreased levels of trophic resolution.

Parasites were disruptive to the persistence, total biomass, and activity of
free livers in food webs. This disruption was roughly proportional to the
fraction of parasites in each web and was made worse by protecting parasites
when they are inside hosts. Parasites generally had very low biomass and
persistence at all levels of parasitism; however, the final activity of
parasites in the webs was commensurate with the free living activity. Parasites
were less disruptive in webs with larger free livers but had much lower final
persistence, biomass, and activity in those webs. Smaller parasites were
slightly more disruptive to the webs and had less biomass, persistence, and
activity than larger parasites.

\section{Future Work}

This work suggests several fruitful avenues of research. First, a deeper
understanding of the structural properties that identify parasites is
warranted. This work did not provide a satisfying ecological interpretation of
the way that the classification trees were able to identify parasites. A
specific understanding of how those properties affected the fit of food web
models and how those properties affect the dynamics of a species are also
lacking.

The inverse niche model is not as effective at modeling parasites as the niche
model is at modeling free livers. This suggests that parasite may require
additional considerations to be accurately captured in a food web model. An
additional trait axis may be able to capture more of the empirical properties
of parasites. Another option may be to assign hosts to parasite based on
phylogenetic similarity (for which trophic similarity could serve as a proxy in
a completely stochastic model). Verification of the importance of the inverted
hierarchy for parasites is necessary; it is possible that controlling the
generalities of parasites without inverting their generality or feeding
hierarchy would result in the same improvements.

The dynamic models also require a deeper analysis of the extinction events to
determine the ecological and mathematical reasons for the disruption caused by
parasites. Of particular interest is the decrease in persistence and biomass
of parasites when they are protected from predation while inside their hosts.
However, a complete understanding of how any species goes extinct in these
simulations is lacking and also bears further investigation. 

\end{bibunit}

\end{document}
