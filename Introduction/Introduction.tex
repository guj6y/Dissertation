\documentclass[../dissertation.tex]{subfiles}
\begin{document}

\begin{bibunit}
%This chapter should give some context to a fairly broad audience.
\chapter{Introduction}

A complete understanding of ecosystems would seem to depend on the number and
diversity of species and interactions included within that understanding.
However, given the spectacular diversity at so many scales e.g., from
physiology to evolution and from bacteria to whales, embarking on a search for
such completeness could condemn ecologists to a Sisyphean task of untangling
Darwin's impenetrable ``entangled bank'' of species and their interactions.
Comprehending this mind boggling complexity of nature has usually
been pursued with what might be termed a divide-and-conquer approach.
Individual processes are observed and combined, for example, birth and death
rates of a species can be combined into simple mathematical models of the
population (exponential growth or decay, \cite*{Malthus1817}). The way that these birth and death
rates vary with that species' population can yield a more realistic model
(logistic model, \cite*{Verhulst1844}) and the way that those rates vary in the presence of other
species' populations can yield even more realistic models (Lotka-Volterra type
models, \cite*{Lotka1920,Volterra1928}). Deciding which processes control species and their interactions is a
challenging task in community ecology.  Once chosen, a first step in developing
mathematical models for population or biomass of entire ecosystems (i.e. the
synthesis of inter- and intraspecies processes) includes understanding how
those interactions are structured.

\section{Empirical Food Webs}

Ecologists face these challenges by specializing into subdisciplines that focus
on different types of interactions within and among organisms. For example,
population ecologists often focus on interactions affecting birth and death
rates while plant ecologists often focus on competition for light, water, and
other nutrients. Ecologists often focus on how energy (measured by, say, nutrients
or biomass) flow through an ecosystem \cite*{Elton1927, MacArthur1955}.
An overall description of this flow for an individual ecosystem is usually
given by assembling a food web of all observed feeding interactions between
multiple species in the ecosystem. These empirical food webs form the basis for
any understanding of how feeding relationships are organized within observed
ecosystems.

Empirical food webs are assembled through a combination of direct observation,
inference (links are known to be present between the species in other areas),
modeling (links exist between similar species), and speculation (a link is
presumed to exist given certain characteristics of the taxa) (e.g.
\cite*{Hechinger2011a}). A significant issue, especially present in historical
food webs, is uneven resolution across trophic levels \cite*{Martinez1993}. Different studies
significantly undersample or aggregate different classes of species, for
example plants, decomposers, or parasites.  The emphasis in food webs is often
on apex predators of economic (e.g. fish exploited by humans in fisheries) or
emotional (e.g. birds or other top predators) value (cf.
\cite*{Yodzis1998,Pikitch2004,Koehn2016}). This uneven resolution and
the disparate foci make generalizations about food web structure somewhat
tenuous and complicate studies of the energy flow through ecosystems.

More recent studies of food webs are characterized by greater resolution across
trophic levels and the sampling of traditionally undersampled species. There
has also been a push for a synthesis of many different types of interactions
between species. Several studies were published in 2011 that accounted for a
potential 21 different types of links \cite*{Hechinger2011a, Mouritsen2011,
Zander2011, Thieltges2011}. Pollination networks and networks of mutualistic
interactions (e.g., frugivory) have also seen a renewed focus
\cite*{Olesen2002,Bascompte2003,Olesen2007,Jordano2003}. The extent to
which these alternative interaction types affect the flow of energy through an
ecosystem is not well understood \cite*{Lafferty2006}. Apart from the direct
effects of parasites on their hosts, they have also been demonstrated to
significantly impact the habitat composition and species diversity through
indirect effects \cite*{Mouritsen2010, Wood2007, Dunn2012}. However, the primary
mechanism of energy flow among species is the consumption of biomass which
makes feeding interactions the backbone of energy flow through ecosystems.

Parasites in particular are hugely diverse and energetically interactive with
many species but have historically gone unsampled in food webs
\cite*{Lafferty2008}. This makes it imperative to discover how this rich
collection of species affects the transfer of energy in food webs. For example,
parasites often have the greatest effects on the populations of the most highly
abundant species \cite*{Gartazar2006, Rossi2005, Vicente2004}. Beyond their
ability to occasionally cause an epidemic and decimate populations, their high
abundance in natural ecosystems hints at more consistent impacts on the
transfer of energy between species \cite*{Whitney2008, Lettini2009}.
Understanding the role of parasites within a backbone of a predator-prey food
web is crucial to assessing their impact on mathematical models of ecosystems.

\section{Food Web Models}

There are relatively few well-resolved food webs in the literature. What webs
there are often have issues. Given the limited quantity and quality of
empirical webs, it is useful to have a good model for how food webs are
assembled. Such a model should be able to generate ensembles of realistic food
webs and infer distributions of simple properties such as generality,
vulnerability, or trophic level \cite*{Williams2000, Williams2008a}.
Distributions of more complicated properties such as robustness or persistence
can also be constructed with additional assumptions about biomass dynamics or
species loss. Food web models can provide us with large collections of highly
controlled food webs that help us better understand ``true'' food webs in
nature.

A key finding of early work in food web models is that the links in food webs
are not randomly distributed among pairs of nodes \cite*{Cohen1990}. Imposing a
hierarchy of species in which species higher in the hierarchy eat species lower
in the hierarchy was found to be an important factor in the organization of
links in food webs \cite*{Cohen1990}. Another important factor is contiguity of
diets, that is, species tend to eat all species within a range of that
hierarchy \cite*{Williams2000}. The fact that such simple rules are able to
produce network structures that are statistically similar to empirical webs
suggests that an important underlying mechanism of food web formation is being
captured.

More complicated food web models have been developed that are able to capture
as many of the observed links as is possible given certain assumptions on the
food web structure \cite*{Williams2010,Williams2011,Allesina2008}. More
mechanistic models based on phylogenetics have also been shown to accurately
capture food web structure \cite*{Cattin2004}. These models have helped
researchers understand the distribution of links between species in food webs,
as well as explain the assumed hierarchy in simpler food web models
\cite*{Williams2011}.
 
Despite these successes, many of these models are tested primarily on webs with
predator-prey and plant-herbivore interactions. The extent to which these
models have been able to reproduce other types of interactions has been little
studied; indeed, important questions remain about whether other interaction
types deserve special treatment or the magnitude of their impact on energy
flows through an ecosystem. For example, there has been much research on
pollination networks but a thorough synthesis of this research with food web
research has yet to appear. A model of parasite interactions based on the
existing assumptions of a hierarchy and contiguity of diets has also been
proposed, but the lack of reliable empirical food webs incorporating well
resolved parasite communities meant that the assessment of its applicability
was limited \cite*{Warren2010}. 

\section{Dynamical Models}

Food webs, whether modeled or empirical, are barely sufficient by themselves to
understand the energy flow through an ecosystem and their static nature means
they can provide very little information about how the food webs respond to
changes in environment. Thus, dynamic models of population or biomass are
important for gaining a thorough knowledge not only of observed ecosystems, but
for understanding how ecosystems are likely to behave in general. A major
challenge in developing mathematical models for population or biomass is in
parameterizing all the interactions that are present in an ecosystem.
Traditional models based on birth, death, and attack rates are virtually
impossible to parameterize given the infeasibility of measuring all the
requisite rates for all species and all interactions between species
\cite*{Yodzis1992}. If such models are developed, their applicability outside of
the ecosystem of interest is limited.

One way forward is to use the results of metabolic theory to parameterize a
mathematical model based on the allometry of metabolic rate. The Allometric
Trophic Network (ATN) model uses the allometry of metabolic rate and
assimilation rates to parameterize a model for an arbitrary number of species
and interactions given only body sizes for each species and a few scaling
constants \cite*{Williams2007, Yodzis1992}. The ATN model has proven moderately
successful at predicting the seasonal dynamics of a temperate lake with very
limited inputs. The addition of system and species-specific modifications
resulted in the seasonal biomass and production being reproduced with a
Bray-Curtis similarities of 0.83 and 0.88, respectively \cite*{Boit2012}.
The ATN model has provided a means of connecting the structure and properties
of food webs to their dynamical stability and ability to produce coexistence
between many species.

However, like the structural models, the ATN has been tested primarily on
predator-prey and herbivorous interactions principally defined by body size
ratios \cite*{Brose2006b, Otto2007, Jonsson2018}. The extent to which the
underlying dynamics of the predator-prey and herbivorous interactions are
modified with the addition of different link types is not well understood (but
see \cite*{Mougi2012} and \cite*{Kefi2016}). Indeed, it is not clear how such
disparate links might be modeled in the ATN framework. For example, pollination
links between plants and their pollinators represent two effects, one feeding
and the other reproductive, that cannot be captured by a simple consumption
interaction.

Parasites pose an interesting problem for the ATN model. On the one hand, they
are simply consumers with small body sizes. On the other hand, they have
traditionally been successfully modeled in epidemiology as pathogens in terms
of their impact on a host population. It is not known whether their effect on
host populations can be well captured by the ATN model using the natural
metabolic scaling suggested by metabolic theory. The incorporation of these
species into the dynamical framework of the ATN model would significantly
extend its predictive power and empirical relevance.

\section{Outline}

We first examine empirical food webs with well-resolved parasite communities
for evidence of unique structural patterns of parasite nodes and of parasite
communities. We also test the robustness of any observed patterns against
decreases in trophic resolution. We then present a series of modifications to a
structural model of parasites and assess the ability of those models to
accurately reproduce properties of empirical webs. We do this at various levels
of trophic resolution. We conclude with a study on the ability of the ATN model
to produce webs with coexisting free liver and parasite communities. We extend
the ATN model to better incorporate elements of the unique life strategy of
parasites.

%
%Ecologists have long sought the infamous "devious strategies" that foster
%stability in the complex ecosystems found in nature \cite*{Mays1972}. Mays's
%analysis was beautiful in its simplicity and devastating in its implications.
%It represented a first step towards generalizing mathematical of the population %of single species to entire communities of species.
%
%A fairly consistent story
%has emerged that when consumers and resources are arranged into feeding
%relationships with consumer-resource body size ratios
%
%With the benefit of 40 years perspective, we can 
%Some more words here.
%
%The structure of food webs - contrary to the random network of interactions
%studied by Mays 40 years ago - display prdictable patterns of interactions. It
%is through the allometric constraints imposed by body sizes that much promising
%research has alleviated the catastrophic predictions of Mays's pioneering work
%\cite*{Brose2006,Otto2007,Berlow2009,Allesina2012}. While these results are
%applicable to many of the interactions found in nature, they certaintly do not
%cover them all. \cite*{Allesina2012}, for instance, found that while
%trophic interactions with consumer-resource body size ratios tended to increase
%web persistence, mutualistic and
%competitive relationships tended to decrease stability. \cite*{Otto2007} found
%that a wide range of body size ratios between the top and intermediate species
%in a tritrophic food chain 
\clearpage
\addcontentsline{toc}{section}{References}
\putbib
\end{bibunit} 

\end{document}
