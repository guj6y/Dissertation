\documentclass[../dissertation.tex]{\subfiles}

\begin{document}
\chapter{Introduction}
Ecologists have long sought the infamous "devious strategies" that foster
stability in the complex ecosystems found in nature \cite{Mays1972}. Mays's
analysis was beautiful in its simplicity and devastating in its implications.
It represented a first step towards generalizing mathematical of the population
of single species to entire communities of species.

A fairly consistent story
has emerged that when consumers and resources are arranged into feeding
relationships with consumer-resource body size ratios

With the benefit of 40 years perspective, we can 
Some more words here.

The structure of food webs - contrary to the random network of interactions
studied by Mays 40 years ago - display prdictable patterns of interactions. It
is through the allometric constraints imposed by body sizes that much promising
research has alleviated the catastrophic predictions of Mays's pioneering work
\cite{Brose2006,Otto2007,Berlow2009,Allesina2012}. While these results are
applicable to many of the interactions found in nature, they certaintly do not
cover them all. \cite{Allesina2012}, for instance, found that while
trophic interactions with consumer-resource body size ratios tended to increase
web persistence, mutualistic and
competitive relationships tended to decrease stability. \cite{Otto2007} found
that a wide range of body size ratios between the top and intermediate species
in a tritrophic food chain 
\end{document}
