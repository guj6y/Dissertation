\documentclass[11pt,letterpaper,twoside,notitlepage]{report}

\usepackage{subfiles}

\usepackage{tikz,
pgfplots,
graphicx,
caption,
multirow,
appendix,
pgfplotstable,
cite,
booktabs,
xcolor,
soul,
geometry,
amsmath,
setspace,
lineno,
tabularx,
rotating,
amsfonts,
pdfpages,
textcomp
}
\usepackage[sectionbib,globalcitecopy]{bibunits}

\renewcommand{\bibname}{References}
\bibliographystyle{acm}
%\linenumbers 
\usepackage[labelformat=simple]{subcaption}
\renewcommand\thesubfigure{(\alph{subfigure})}
\usepackage[hidelinks]{hyperref}

\usepackage[author=]{fixme}
\fxuselayouts{margin}
\fxusetheme{color}
\definecolor{targetbkg}{RGB}{158,241,253}
\sethlcolor{targetbkg}
\makeatletter
\newcommand\FXTargetLayoutHigh[2]{\@fxuseface{target}\hl{#2}}
\makeatother
\FXRegisterTargetLayout{high}{\FXTargetLayoutHigh}
\fxusetargetlayout{high}

%\input{header.tex}
\newcommand{\BSR}{BSR}
\newcommand{\aBSR}{mean BSR}


%Commands to work with tables
\newcommand{\getEl}[2]{\pgfplotstablegetelem{#1}{#2}\of}


\newcommand{\diff}[3]{%
\pgfmathparse{#2 - #1}%
\edef#3{\pgfmathresult}%
}

\newcommand{\perChange}[3]{%
    \pgfmathparse{100*abs((#2- #1)/#1)}%
\edef#3{\pgfmathresult}%
}

\newcommand{\diffCol}[3]{%
    \getEl{1}{#2}#1%
    \edef\valI{\pgfplotsretval}%
    \getEl{9}{#2}#1%
    \edef\valII{\pgfplotsretval}%
    \diff{\valI}{\valII}{\diffVal}%
    \edef#3{\diffVal}%
}

\newcommand{\perChangeCol}[3]{%
    \getEl{1}{#2}#1%
    \edef\valI{\pgfplotsretval} %
    \getEl{9}{#2}#1%
    \edef\valII{\pgfplotsretval}%
    \perChange{\valI}{\valII}{\perChangeVal}%
    \edef#3{\perChangeVal}%
}

\newcommand{\perChangePrint}[3]{%
    \getEl{1}{#3}#2%
    \edef\valI{\pgfplotsretval}%
    \getEl{9}{#3}#2%
    \edef\valII{\pgfplotsretval}%
    \perChange{\valI}{\valII}{\perChangeVal}%
    \pn[#1]{\perChangeVal}%
}

\newcommand{\pn}{\pgfmathprintnumber}

\newcommand{\printFromTab}[4]{\pgfplotstablegetelem{#1}{#2}\of#3\pn[#4]{\pgfplotsretval}}

\newcommand\DissertationDir{/home/nkappler/Research/Dissertation/}
\doublespacing
\newcommand\graphicsExtension{pdf}

\defaultbibliography{Bib_green}
\defaultbibliographystyle{plain}

\title{Effect of Parasites on the Structure and Dynamics of Food Webs}
\author{Nick Kappler}

\begin{document}

%Preamble; toc; titlepage, etc.
\bibliographyunit[\chapter]
\subfile{titlepage.tex}

    \newpage
    \thispagestyle{plain}
    \setcounter{page}{2}

    \includepdf[pagecommand = {}]{apvl-wm.pdf}

\includepdf[pagecommand = {}]{Statement.pdf} 


\begin{center}
{\Large\textbf{Acknowledgments}} 
\end{center} 

I would like to thank all the people who helped to make this possible. This
includes old friends who encouraged me, new friends who helped ease the
adjustment to graduate school and managed to make it an enjoyable experience.
My parents and my parents-in-law for all the support that they've given over
the years. I would like to thank my Jim Cushing and Ryan Gutenkunst for bearing
with me these last few months of writing and for providing helpful comments on
the dissertation. I would like to thank Neo Martinez for taking me on as a
student and providing a positive and enriching lab setting. Finally, I want to
thank my daughter, Lily, and wife, Laura, for always keeping me grounded and
for their patience through all the long nights. And for putting up with a
grumpy graduate student for the last three years and a nervous wreck for the
last four months.

\addcontentsline{toc}{chapter}{Abstract}
 
\tableofcontents
\listoffigures

\addcontentsline{toc}{chapter}{List of Figures}

\listoftables

\addcontentsline{toc}{chapter}{List of Tables}

\newpage

\begin{abstract}
    
    Food web theory continues to be an important avenue of research in Ecology.
    Accounting for the diverse relationships between species in an ecosystem is
    fundamental to understanding how that system responds to the changing world
    we live in. Incorporating ever more diverse sets of species is vital to
    that goal and parasites are an important group of species that have yet to
    be wholly integrated with food web research. This dissertation provides
    three key insights into the roles that parasites play in food webs. First,
    by examining several empirical food webs, we show that though there are
    reliable structural differences between parasites and free living (that is,
    non-parasite) species at both the community and individual level, there are
    more similarities than differences. Second, we propose an extension to an
    existing structural model with parasites that is able to accurately capture
    structural properties of empirical food webs with parasites better than the
    Niche Model does. Finally, we examine the feasibility of incorporating
    parasites in existing allometric trophic network theory. We find that even
    when accounting for concomittant predation and a host-as-refuge effect,
    parasites are very disruptive when their metabolic rates are defined by
    standard metabolic theory.

\end{abstract}

\subfile{Introduction/Introduction.tex}
\newpage

\subfile{Chapter2/Chapter2.tex}
\newpage

\subfile{Chapter3/Chapter3.tex}
\newpage

\subfile{Chapter4/Chapter4.tex}
\newpage

\subfile{Conclusion/Conclusion.tex}

\putbib

\end{document}
