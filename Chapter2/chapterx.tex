\documentclass{article}

\title{}

\begin{document}
\maketitle

\chapter{Chapter 2: Empirical Food Webs and the Parasitic Niche}

\section{Abstract}

\section{Introduction}

\section{Data} We analyzed 6 speciose food webs with well-resolved parasite
communities. See table \ref{tab:foodWebSummary} for a summary of the basic
structural properties in each of the six webs. All six food webs represent
coastal ecosystems. Three are located on the pacific coast of southern
California and Mexico \cite{Hechinger2011}; two are located on the north sea
\cite{Thieltges2005,Zander2011}; one is located in New Zealand
\cite{Mouritsen2006}. 

These food webs have been the subject of studies of the effect of parasites on
global network properties as well as their robustness to species removal
\cite{Dunne2013,Lafferty2012}. This study extends previous work by attempting
to elucidate unique features of parasitic niches in these food webs with the
aim of generating a set of rules that identify parasites strictly by their
local topological properties within a food web.

\section{Methods} We first constructed trophic webs from each of the 6 raw food
webs. The raw food web data differentiates between 21 types of interspecies
interactions. We distinguished two different link types: (1) free-living
consumption and (2) parasitism. See table \ref{tab:foodWebLinks} for a summary
of link types in the raw data and how each type was classified in our
construction of the trophic food web. We classified a species as a parasite if
it was the consumer node in a parasitism link; all other consumers were
classified as free-livers. We then aggregated nodes in the raw web with
identical consumer and resource sets into trophic nodes. 

\subsection{Nodal Properties of Parasites} We calculated local properties of
nodes in an attempt to find a structural fingerprint of parasitism. The
properties calculated include traditional local properties of ecological
interest (vulnerability, generality, and trophic level) and network theoretic
properties not commonly found in ecological studies (see below). See table
\ref{tab:localProperties} for a summary of these network properties.
\fxnote{References for these properties.} 

\subsubsection{Betweenness centralities} Betweenness centrality is a measure of
how often a particular node appears in shortest paths between each other pair
of nodes. This measure was developed in the context of information flow through
social networks; nodes with a high betweenness are likely to be gatekeepers
between highly connected components. The original measure as developed in
\cite{Freeman1977,Anthonisse1971} of a node, $v$, is

\begin{equation} 
    C_B(v) = \sum_{s\neq\v\neq\t\inV}\frac{\sigma_{st}(v)}{\sigma_{st}}
\label{eq:BetweennessCentrality} 
\end{equation}

where $\sigma_{st}(v)$ is the number of shortest paths between node $s$ and $t$
that pass through $v$, and $\sigma_{st}$ is the total number of shortest paths
between nodes $s$ and $t$.

We offer a slight refinement of betweenness centrality by only counting
shortest paths between a basal species and a consumer species. The ecological
betweenness centrality, $C_EB$ of a node $k$, is given by:

\begin{equation} 
    C_{EB}(v) = \sum_{s\in\text{bas},t\in\text{con},t\neq v}\frac{\sigma_{ij}(v)}{\sigma_{ij})
\label{eq:EcologicalBetweennessCentrality} 
\end{equation}

where $\sigma_{st}$ and $\sigma_{st}(v)$ are as in
\ref{eq:BetweennessCentrality}. We believe this may be of greater ecological
significance as the shortest path to a basal is likely to be much more
energetically important for a species than the shortest path between two
species at similar trophic levels.

\subsubsection{Mean Generality of Consumers} This property does exactly what it
says: it is the average normalized generality of all the consumers of a
particular node. This aims to measure the importance of a node in the diet of
its predators. If $\mathcal{C}(i)$ is the set of consumers of $i$ and $g_j$ is
the generality of node $j$, then the mean generality of consumers of node $i$,
$g_c(i)$ is given by

\begin{equation} 
    g_c(i) = \frac{1}{|\mathcal{C}(i)|}\sum_{j\in\mathcal{C}(i)}g_j
\label{eq:MeanGeneralityConsumers} 
\end{equation}

\subsubsection{Mean Vulnerability of Resources} This property also does exactly
what it says it advertises: it is the average normalized vulnerability of all
the resources of a particular node. This aims to measure the topological
importance of a node on its resources. If $\mathcal{R}(i)$ is the set of
resources of $i$ and $v_j$ is the vulnerability of node $j$, then the mean
vulnerability of resources of node $i$, $v_r(i)$ is given by

\begin{equation} 
    v_r(i) = \frac{1}{|\mathcal{R}(i)|}\sum_{j\in\mathcal{R}(i)}v_j
\label{eq:MeanVulnerabilityResources}  
\end{equation}

\subsubsection{An Eigenvalue Based Measure of Centrality} The PageRank measure
achieved fame as one of the metrics behind the popular Google search engine. We
follow \cite{Allesina2009} in a modification of the PageRank algorithm to
create an eigenvalue-based centrality measure suitable for food webs.  Roughly
speaking, this measure quantifies of the importance of each node in terms of
nutrient flow through the ecosystem.

\subsubsection{Normalization} The above measures can be highly dependent on
network size and complexity. Before comparing parasite and free-living nodes,
we normalize each metric by dividing by the average value of all consumers.

\subsection{Agglomeration of Webs} We were also interested in how food web
resolution affected the above measures of centrality. For each of the 6
empirical webs, we used the unweighted pair group method with arithmetic mean
(UPGMA, \cite{Sokal1958}) with the Jaccard distance to group similar nodes as
in \cite{Martinez1991}. The Jaccard Distance, \cite{Jaccard1908} was calculated
as 

\begin{equation} 
    d_J(s,t) = 1-\frac{|\mathcal{N}(s)\cap \mathcal{N}(t)|}{\frac{|\mathcal{N}(s)\cup \mathcal{N}(t)|}
\label{eq:JaccardDistance} 
\end{equation} 

where $\mathcal{N}(v)$ is the set of all neighbors (i.e. all predators and all
prey) of node $v$. We define the distance between two clusters of nodes using
average linkage clustering: for two clusters $\mathcal{A}$ and $\mathcal{B}$,
define $d_J(\mathcal{A},\mathcal{B})$ as

\begin{equation} 
    d_J(\mathcal{A},\mathcal{B}) = \frac{1}{|\mathcal{A}|\cdot|\mathcal{B}|}\sum_{s\in\mathcal{A}}\sum_{t\in\mathcal{B}}d_J(s,t)
\label{eq:averageLinkage} 
\end{equation}

The result of the clustering algorithm was a sequence of clusters of species;
in order to define a web at each level of aggregation, we also needed to define
feeding relationships between the clusters. We used the "maximum linkage
criterion" as defined in \cite{Martinez1991}. The six webs studied here had
much lower connectances than the Little Rock Lake food web, so more stringent
linkage criteria (i.e. the "mean linkage criterion") quickly resulted in
disconnected webs. 

For each food web defined along the clustering sequence, we calculated the
aforementioned nodal and global properties. 

\subsubsection{Breaking Ties} The sequence of clusters resulting from
agglomerative clustering algorithms are non-deterministic in the presence of
ties in the distance between pairs of clusters. At each stage, we first
identified all potential mergers between clusters. We then label the smaller
cluster the \textit{satellite} and the larger cluster the \textit{sink} for
each merger; if the two clusters for a given merger were the same size, each
cluster had equal probability of being assigned as the \textit{sink}. After
identifying the satellite and the sink, we model the merger of two clusters as
the satellite being subsumed into the sink. After identifying the satellites
and the sinks, we merge the smallest satellites into the smallest sinks. When a
satellite merges into a sink, it is ineligible for any further mergers. The
sinks, however, remain eligible for further mergers until they are subsumed
into a larger sink. Any ties between mergers with the same size satellite and
same size sink were broken at random. We used the tie-breaker to encourage a
more even distribution of nodes among clusters. \fxnote{Explain the basic thing
    I was worried about here? But, it's not really part of my THESIS so I don't
want to belabor it.} 

\section{Results} 
The parasite average plotted against the free-liver average
for each empirical trophic web is given in figure
\ref{fig:InitialNodalProperties}. There was little evidence of consistent
differences between parasites and free-livers on average for generality,
clustering coefficient or classic betweenness centrality. 

\subsection{Initial Webs}

\subsection{Agglomerated Webs}

    
\section{Conclusions} Conclusions go here.

\end{document}
